\documentclass[11pt]{article}
\usepackage[english]{babel} 
\usepackage[utf8]{inputenc}

\usepackage{multirow}
\usepackage{tabularx} 
\usepackage{forloop} 
\usepackage{graphicx}
\usepackage{lipsum}
\usepackage{listings}
\usepackage{cite}
\usepackage{color}
\usepackage{datetime}
\usepackage{advdate}
\usepackage{mathptmx}
\usepackage{amsmath} 
\usepackage{caption}
\usepackage{subcaption}


\begin{document}
\setlength\parindent{0pt}
\title{Mandatry Assignment no.2}
\author{Lars Magnus Valnes \\  
INF5620 
}
\maketitle

%\tableofcontents
%\clearpage
\section{Introduction}
We will in this text go through the derivation required to complete exercise 13: Compare discretizations of a Neumann condition. In this exercise we will look at different ways to handle spacial Neumann conditions for the wave equation with variational wave velocity. 
The partial differential equation is given as
\begin{equation}
u_{tt}= \partial_x \left( q(x) u_{x} \right)+ f(x,t),
\end{equation} 
with $q(x)$ denotes the wave velocity squared, i.e. $q\left( x\right) =c\sp{2} \left( x\right) $ and $f$ as the source term.
\\
The discretization scheme can be expressed in compact notation as 
\begin{equation}
\left[ D_tD_t u \right]_{i}\sp{n} = \left[D_x \left(q(x) D_x u \right)\right]_{i}\sp{n} + \left[f\right]_{i}\sp{n},
\label{Eq::compact_notation}
\end{equation}
where $n$ is the temporal discertization and $i$ is the spacial discertization.
Writing out the compact notation in Eq.\ref{Eq::compact_notation} with the centered difference scheme we obtain the expression 
\begin{equation}
\frac{u\sp{n+1}_i-2*u\sp{n}_i+u\sp{n-1}_i    }{\left(\Delta t\right)\sp{2}}  =  \frac{ q_{i+\sp{2}/_{2}} \left( u\sp{n}_{i+1} -u\sp{n}_{i} \right) -  q_{i-\sp{2}/_{2}}  \left( u\sp{n}_{i} -u\sp{n}_{i-1} \right)}{\left(\Delta x\right)\sp{2}} + f\sp{n}_i .
\label{Eq::dis_exp}
\end{equation}
Here the notations $\Delta t$ and $\Delta x$ denotes the temporal and spacial discretization lengths.
Rearranging Eq.\ref{Eq::dis_exp} will give an explicit scheme like 
\begin{equation}
u\sp{n+1}_i  =  2u\sp{n}_i-u\sp{n-1}_i + C\sp{2} \left( q_{i+\sp{2}/_{2}} \left( u\sp{n}_{i+1} -u\sp{n}_{i} \right) -  q_{i-\sp{2}/_{2}}  \left( u\sp{n}_{i} -u\sp{n}_{i-1} \right) \right) +  \left(\Delta t\right)\sp{2}f\sp{n}_i ,
\label{Eq::explicit_scheme}
\end{equation}
The notation of $C$ will be used throughout this assignment paper, and defined as  $C = \frac{\Delta t}{\Delta x}$. 
\section{Discretization}

\subsection{Spacial discretization}
The spacial discretization divides the line interval $\left[0,L \right]$ into $N_x$ parts. Thus the end-points are given at $i=0$ and $i=N_x$, and the spacial step length is given by
\begin{equation}
\Delta x = \frac{L}{N_x}
\end{equation}
The handling of the end-points will detailed in section \ref{sec::Neum}. 
\subsection{Temporal discretization}
The temporal discretization divides the time interval $\left[0,T \right]$ into $N_t$ parts. Thus the end-points are given at $n=0$ and $n=N_t$, and the temporal step length us given by
\begin{equation}
\Delta t = \frac{T}{N_t}
\end{equation}
We will handle the initial time step,i.e. $n=1$ using the compact notation 
\begin{equation}
\left[ D_{2t} u\right]\sp{n}_i = 0 .
\end{equation}
which gives $u\sp{-1}_i = u\sp{1}_i$. 
Thus the expilict scheme in Eq. \ref{Eq::explicit_scheme} becomes 
\begin{equation}
u\sp{1}_i  =  u\sp{0}_i + \frac{1}{2}C\sp{2} \left( q_{i+\sp{2}/_{2}} \left( u\sp{0}_{i+1} -u\sp{0}_{i} \right) -  q_{i-\sp{2}/_{2}}  \left( u\sp{0}_{i} -u\sp{0}_{i-1} \right) \right) + \frac{1}{2}\left(\Delta t\right)\sp{2}f\sp{0}_i ,
\label{Eq::inital_time_step}
\end{equation}


\subsection{The arithmetic mean}
The determination of $q_{i+\sp{1}/_2}$ and $q_{i-\sp{1}/_2}$ can easily be done if $q$ is a known function. In this assignment $q$ is a defined function, thus the evaluation can be written as
\begin{equation}
q_{i\pm \sp{1}/_2} = q\left( x_{i} \pm  \frac{\Delta x}{2}\right)  
\label{Eq::mean}
\end{equation}


\section{Neumann discretization}
\label{sec::Neum}
We start Neumann discretization expressed in compact notation at the spacial boundaries, (i.e. $x=0$ and $x =L $ ) as
\begin{equation}
\left[ D_{2x} u\right]\sp{n}_i = 0 .
\label{Eq::Neumann_spacial_dis}
\end{equation}
This is equivalent to
\begin{equation}
\frac{u\sp{n}_{i+1} - u\sp{n}_{i-1}}{2\Delta x} = 0  ,
\end{equation}
which implies that $u\sp{n}_{i+1} = u\sp{n}_{i-1}$. This can be inserted into Eq.\ref{Eq::explicit_scheme} to obtain
\begin{equation}
u\sp{n+1}_i  =  2u\sp{n}_i-u\sp{n-1}_i + C\sp{2} \left(\left( q_{i+\sp{1}/_{2}} +  q_{i-\sp{1}/_{2}} \right) \left( u\sp{n}_{i-1} -u\sp{n}_{i} \right) \right) +  \left(\Delta t\right)\sp{2}f\sp{n}_i .
\end{equation}
The method given in Eq.\ref{Eq::Neumann_spacial_dis} can also be used for $q$, which results in $q\sp{n}_{i+\sp{1}/_{2}} =  q\sp{n}_{i-\sp{1}/_{2}}$.
This gives the explicit scheme at the spacial boundaries as
\begin{equation}
\begin{aligned}
u\sp{n+1}_i  =  2u\sp{n}_i-u\sp{n-1}_i + C\sp{2}2 q_{i+\sp{1}/_{2}}\left(  u\sp{n}_{i+1} -u\sp{n}_{i} \right) -\left(\Delta t\right)\sp{2}f\sp{n}_i &\quad \text{for} \quad i=0 \\
u\sp{n+1}_i  =  2u\sp{n}_i-u\sp{n-1}_i + C\sp{2} 2 q_{i-\sp{1}/_{2}}\left(  u\sp{n}_{i} -u\sp{n}_{i-1} \right) -\left(\Delta t\right)\sp{2}f\sp{n}_i  &\quad \text{for} \quad i=N_x
\end{aligned}
\end{equation}
with $q\sp{n}_{i+\sp{1}/_{2}}$ and $q\sp{n}_{1-\sp{1}/_{2}}$ determined using Eq.\ref{Eq::mean}.
\\\\
Another method is to approximate the expression $ q_{i+\sp{1}/_{2}} +  q_{i-\sp{1}/_{2}}$ through Taylor expansion, which yields
\begin{equation}
q_{i+\sp{2}/_{2}} + q_{i-\sp{2}/_{2}} = q_i + q_i - \partial_x q_i\frac{\Delta x}{2} +  \partial_x q_i\frac{\Delta x}{2}  + O\left( \Delta x\sp{2}\right) \approx 2q_i .
\end{equation}
Hence Eq.\ref{Eq::explicit_scheme} can be written as  
\begin{equation}
\begin{aligned}
u\sp{n+1}_i  =  2u\sp{n}_i-u\sp{n-1}_i + C\sp{2}  2q_{i} \left(  u\sp{n}_{i+1} -u\sp{n}_{i} \right) -\left(\Delta t\right)\sp{2}f\sp{n}_i &\quad \text{for} \quad i=0 \\
u\sp{n+1}_i  =  2u\sp{n}_i-u\sp{n-1}_i + C\sp{2} 2q_{i} \left(  u\sp{n}_{i} -u\sp{n}_{i-1} \right) -\left(\Delta t\right)\sp{2}f\sp{n}_i  &\quad \text{for} \quad i=N_x
\end{aligned}
\end{equation}

For the remaining parts in this assignment, we will select 
\begin{equation}
q\left( x \right) = 1 + \text{cos}\left(k x \right) ,
\end{equation}
with $k= \pi/_{L}$ . We have now a function of $q$ that is symmetric around the spacial  boundaries, i.e  $x=0$ and $x=L$. This means that $\partial_x q(x) = 0$, thus the approximation $q_{1\pm\sp{1}/_2}$ as
\begin{equation}
q_{1\pm\sp{1}/_2} =  q_i \pm \partial_x q_i\frac{\Delta x}{2} + O\left( \Delta x\sp{2}\right) \approx q_i .
\label{Eq::sym_approx}
\end{equation}

%Thus the approximation of $ q\sp{n}_{i+\sp{1}/_{2}} $ at the spacial boundaries becomes
%\begin{equation}
% q\sp{n}_{i+\sp{1}/_{2}}  =  q\sp{n}_i - \partial_x q\sp{n}_i\frac{\Delta x}{2} +  \partial_x\sp{2} q\sp{n}_i\left( \frac{\Delta x}{2}\right)\sp{2} \approx  q\sp{n}_{i} ,
%\end{equation}
%since that symmtery proerty gives $\partial_x q(x) = 0$.
%\begin{equation}
%\begin{aligned}
%u\sp{n+1}_i  =  2u\sp{n}_i-u\sp{n-1}_i + C\sp{2} 2 q\sp{n}_i\left(  u\sp{n}_{i+1} -u\sp{n}_{i} \right) -\left(\Delta t\right)\sp{2}f\sp{n}_i &\quad \text{for} \quad i=0 \\
%u\sp{n+1}_i  =  2u\sp{n}_i-u\sp{n-1}_i + C\sp{2} 2 q\sp{n}_i\left(  u\sp{n}_{i} -u\sp{n}_{i-1} \right) -\left(\Delta t\right)\sp{2}f\sp{n}_i  &\quad \text{for} \quad i=N_x
%\end{aligned}
%\end{equation}
The next discretization method is the one-sided approximation yielding 
\begin{equation}
\begin{aligned}
  u\sp{n}_{i} -  u\sp{n}_{i-1} = 0 &\quad \text{for} \quad i = 0\\
  u\sp{n}_{i+1} -  u\sp{n}_{i} = 0 &\quad \text{for} \quad i = N_x
\end{aligned}
\label{Eq::one_sided}
\end{equation}
Using Eq.\ref{Eq::one_sided} combined with Eq.\ref{Eq::explicit_scheme} and Eq.\ref{Eq::sym_approx} will results in the explicit scheme 
\begin{equation}
\begin{aligned}
u\sp{n+1}_i  =  2u\sp{n}_i-u\sp{n-1}_i +  C\sp{2}  q_{i} \left( u\sp{n}_{i+1} -u\sp{n}_{i} \right) +  \left(\Delta t\right)\sp{2}f\sp{n}_i , &\quad \text{for} \quad i = 0 \\
u\sp{n+1}_i  =  2u\sp{n}_i-u\sp{n-1}_i +C\sp{2}q_{i} \left( u\sp{n}_{i-1} -u\sp{n}_{i} \right)  +  \left(\Delta t\right)\sp{2}f\sp{n}_i , &\quad \text{for} \quad i = N_x .
\end{aligned}
\end{equation}
In the last Neumann discretization technique, we will use the compact notation 
\begin{equation}
\left[ D_t D_t u \right]\sp{n}_i = \frac{1}{\Delta x}\left(\left[q D_x u \right]_{i+\sp{1}/_2}\sp{n}  -\left[q D_x u \right]_{i-\sp{1}/_2}\sp{n} \right) + \left[ f \right]\sp{n}_i.
\label{Eq::last_compact}
\end{equation}
We will place the boundary at $x_{i+\sp{1}/_2}$ with $i=N_x$ and at  $x_{i-\sp{1}/_2}$ with $i=0$, yielding 
\begin{equation}
\begin{aligned}
  \left[q D_x u \right]_{i+\sp{1}/_2}\sp{n} &= 0 \quad \text{for} \quad i = N_x \\
  \left[q D_x u \right]_{i-\sp{1}/_2}\sp{n} &= 0 \quad \text{for} \quad i = 0
\end{aligned}.
\label{Eq::fourth_approx}
\end{equation}
We can insert Eq.\ref{Eq::fourth_approx} into Eq.\ref{Eq::last_compact} and obtain the explicit scheme for the end-points as
\begin{equation}
\begin{aligned}
u\sp{n+1}_i  =  2u\sp{n}_i-u\sp{n-1}_i + C\sp{2}q_{i+\sp{2}/_{2}}  \left( u\sp{n}_{i+1} -u\sp{n}_{i} \right)  +  \left(\Delta t\right)\sp{2}f\sp{n}_i & \quad \text{for} \quad i = 0 \\
 u\sp{n+1}_i  =  2u\sp{n}_i-u\sp{n-1}_i + C\sp{2}q_{i-\sp{2}/_{2}}  \left( u\sp{n}_{i-1} -u\sp{n}_{i} \right)  +  \left(\Delta t\right)\sp{2}f\sp{n}_i & \quad \text{for} \quad i+1/2  N_x 				
\end{aligned}
\label{Eq::Last_scheme}
\end{equation}
The boundary placement means that Eq.\ref{Eq::sym_approx} is not valid.




\end{document}